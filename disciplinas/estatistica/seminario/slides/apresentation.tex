\documentclass{beamer}
\usepackage{tipa}
\usepackage{graphicx}
\usepackage{dsfont}
\usepackage{amssymb}
\usetheme{Copenhagen}

\title{René Descartes e Claudio Landim}
\author{Lorran de Araújo Durães Soares}
\institute{Licenciatura em Matemática\\Universidade Estadual de Montes Claros\\Campus São Francisco - MG}

\begin{document}


\begin{frame}
\titlepage
\end{frame}

%%% René Descartes %%%


\begin{frame}{René Descartes}
    \begin{block}{Introdução}
        \begin{itemize}
            \item Importante filósofo e matemático, além de ter deixado significativas contribuições para a Física. 
            \item Seu método filosófico introduziu um pensamento mais exato no campo da Filosofia — o que o fez ser considerado o primeiro filósofo da vertente racionalista e colocou-o em posição de destaque para a constituição do pensamento moderno. 
            \item Graças ao rigor dos estudos matemáticos de Descartes e à criação de seu plano de coordenadas, é possível hoje estabelecer os estudos da geometria analítica e da geometria espacial com maior precisão.
        \end{itemize}
    \end{block}
\end{frame}


% \begin{frame}{}
%     \begin{center}
% Imagem 1: René Descartes \\
%         \includegraphics[width=6cm]{descartes.jpg}
%     \end{center}
% \end{frame} 


\begin{frame}
    \begin{block}{Biografia}
        \begin{itemize}
            \item Nascido em Haye, província francesa, em 1596. 
            \item Órfão de mãe quase um ano após o seu nascimento, cresceu sob os cuidados do pai e de uma ama. 
            \item Seu pai era funcionário público de Haye e providenciou uma educação de elite para o filho, que, desde cedo, teve contato com a Filosofia, com a Astronomia e com a Matemática.
            \item Descartes estudou no colégio Royal Henry Le Grand, seminário dirigido por jesuítas no Castelo de La Flèche. Aos 19 anos terminou o seminário. Quando terminou o curso básico, ingressou no curso de Direito na Universidade de Poitiers. 
            \item Aos 22 anos de idade, tornou-se bacharel em Direito, mas nunca exerceu a advocacia ou se envolveu com a carreira jurídica.
        \end{itemize}
    \end{block}  
\end{frame}


\begin{frame}
    \begin{block}{}
        \begin{itemize}
            \item Posteriormente, envolveu-se com a carreita militar.
            \item 1 ano sendo soldado.
            \item Quase toda vida foi conselheiro e estrategista militar, atividade que era secundária, pois sua dedicação estava voltada à Filosofia e à Matemática.
            \item Era sempre reservado e evitava convívio social, preferendo ficar sozinho e cair em seus profundos pensamentos, tendo deixado poucos vestígios de sua vida pessoal.
        \end{itemize}
    \end{block}  
\end{frame}


\begin{frame}
    \begin{block}{}
        \begin{itemize}
            \item Também com 22 anos começou a estudar Matemática e viu-se fascinado pela exatidão daquela ciência. 
            \item Aos 33 anos, Descartes havia escrito um livro intitulado Tratado sobre o mundo. O filósofo optou por não publicar o manuscrito sobre ciência natural, que defendia uma tese heliocêntrica, em decorrência do processo e condenação vividos por Galileu Galilei.
            \item Em 1637, Descartes publica Discurso do método, sua mais importante obra filosófica, e, em 1641, publica Meditações metafísicas, outra grande obra de sua autoria. 
            \item Em 1649, o pensador francês aceita o convite da rainha Cristina, da Suécia, e parte para ser conselheiro dela. O rigoroso inverno sueco causa uma forte pneumonia em Descartes, que morre em 1650. 
            \item Em 1663, a Igreja Católica proíbe alguns de seus livros, em especial Meditações metafísicas, por adentrar em assuntos teológicos.      
        \end{itemize}
    \end{block}  
\end{frame}


\begin{frame}
    \begin{block}{Descartes e a Filosofia}
        \begin{itemize}
            \item Descartes revolucionou o pensamento filosófico moderno. Suas contribuições deram origem à tradição racionalista que se baseia no entendimento de que o conhecimento racional é inato ao ser humano. 
            \item O Racionalismo é uma corrente filosófica que traz como argumento a noção de que a razão é a única forma que o ser humano tem de alcançar o verdadeiro conhecimento por completo. 
            \item Assim como Platão, o filósofo francês concebeu o ser humano como um ser composto por uma dualidade psicofísica, isto é, por uma mente ou alma (psique) e por um corpo.
            
        \end{itemize}
    \end{block}  
\end{frame}

\begin{frame}
    \begin{block}{}
        \begin{itemize}
            \item Esses elementos são designados por Descartes como res cogitans (coisa pensante) e res extensa (coisa extensa). Nessa concepção, a alma ou mente (coisa pensante) é o atributo maior do ser humano e o seu corpo (coisa extensa) é a extensão da alma. O corpo depende da alma para viver do mesmo modo que a alma depende do corpo para habitar o mundo.
            \item Descartes observou que os seus professores de Matemática tinham um método preciso e exato para chegarem a conclusões de raciocínios, o que os garantia certeza e não gerava controvérsias, ao passo que os seus professores de Filosofia envolviam-se em querelas por utilizarem métodos próprios e diferentes. Para Descartes, não era possível firmar uma Filosofia sólida num terreno movediço, fazendo-se necessário estabelecer um método preciso para a Filosofia. 
        \end{itemize}
    \end{block}
\end{frame}


\begin{frame}{}
    \begin{alertblock}{Principais ideias}
        \begin{itemize}
            \item A razão é inata ao ser humano, ou seja, nós já nascemos com as ideias racionais embutidas em nosso intelecto. O que diferencia a inteligência de uns e de outros é a maneira como utilizamos a nossa inteligência.
            \item O conhecimento deve ser claro e distinto. Tudo aquilo que gera dúvida deve ser afastado do âmbito do conhecimento verdadeiro.
            \item O conhecimento filosófico deve ser consolidado por um método que garanta a confiabilidade do que se conhece.
        \end{itemize}    
    \end{alertblock}
\end{frame}


\begin{frame}{}
    \begin{block}{Racionalismo}
        \begin{itemize}
            \item O pensamento racionalista de Descartes toma por base a ideia de que o conhecimento fornecido por outras fontes que não a razão pode ser enganosa. Isso implica que somente o conhecimento racional, fruto das deduções, é claro e distinto. Somente o processo dedutivo (utilizado por excelência na Matemática) pode ser adotado como meio seguro e único para evidenciar aquilo que é conhecido.
    
        \end{itemize}    
    \end{block}
\end{frame}


\begin{frame}
    \begin{block}{}
    Para fundamentar sua teoria, o filósofo elabora um método baseado, primeiramente, na dúvida metódica e hiperbólica.As regras para o método são as seguintes:
        \begin{itemize}
            \item Evidência: Jamais aceitar como verdadeiro algo duvidoso.
            \item Análise: Ao enfrentar um problema filosófico, dividi-lo em partes, para facilitar a sua compreensão.
            \item Síntese: Sempre começar resolvendo os problemas menores, as partes menos complexas, partindo então rumo aos problemas maiores, pois a junção das múltiplas partes pode resolver ou fornecer pistas para a resolução do problema como um todo.
            \item Enumeração: Enumerar todas as partes fracionadas e revisar cada etapa assim que finalizada, pois isso facilita a identificação de erros.
        \end{itemize}
O método cartesiano para a Filosofia fornece elementos para o desenvolvimento de posteriores métodos científicos mais avançados.
    \end{block}
\end{frame}


\begin{frame}
    \begin{block}{Cogito}
O método cartesiano e a sua dúvida metódica e hiperbólica fizeram-no chegar ao cogito, o primeiro conhecimento estritamente verdadeiro, obtido por meio da dedução. Estes foram os passos percorridos pelo filósofo para que chegasse ao cogito:
        \begin{itemize}
            \item Eu devo duvidar de tudo para atingir um conhecimento rigoroso.
            \item Ao duvidar de tudo, duvido inclusive de mim mesmo, de minha essência e de minha existência.
            \item Ao duvidar, eu estou pensando.
            \item Se penso, logo eu existo.
        \end{itemize}
O cogito cartesiano foi traduzido para o Português como “penso, logo existo”.
    \end{block}  
\end{frame}


\begin{frame}{} 
    \begin{block}{Descartes e a Matemática}
        \begin{itemize}
            \item René Descartes deve ser considerado um gênio da Matemática, pois relacionou a Álgebra com a Geometria, o resultado desse estudo foi a criação do Plano Cartesiano. Essa fusão resultou na Geometria Analítica. \item Descartes defendia que a Matemática dispunha de conhecimentos técnicos para a evolução de qualquer área de conhecimento.
            \item Descartes utilizou o Plano Cartesiano no intuito de representar planos, retas, curvas e círculos através de equações matemáticas. Os estudos iniciais da Geometria Analítica surgiram com as teorias de René Descartes, que representavam de forma numérica as propriedades geométricas. 
        \end{itemize}
    \end{block} 
\end{frame}


\begin{frame}
    \begin{block}{}
        \begin{itemize}
            \item A criação da Geometria Analítica por Descartes foi fundamental para a criação do Cálculo Diferencial e Integral pelos cientistas Isaac Newton e Leibniz. O Cálculo se dedica ao estudo das taxas de variação de grandezas e a acumulação de quantidades, sendo de grande importância na Física, Biologia e Química.
            \item Além do Cálculo e da Geometria Analítica, os estudos de Descartes permitiram o desenvolvimento da Cartografia, ciência responsável pelos aspectos matemáticos ligados à construção de mapas.
        \end{itemize}
    \end{block}
\end{frame}

\begin{frame}
    \begin{alertblock}{Frases}
        \begin{itemize}
            \item “Penso, logo existo.”
            \item “O bom senso é a coisa do mundo melhor partilhada.”
            \item “Muitas vezes as coisas que me pareceram verdadeiras, quando comecei a concebê-las, tornaram-se falsas, quando quis colocá-las sobre o papel.”
            \item “Não basta termos um bom espírito, o mais importante é aplicá-lo bem.”
        \end{itemize}
    \end{alertblock}  
\end{frame}


%%% Claudio Landim %%%


\begin{frame}{Claudio Landim}
    \begin{block}{Biografia}
        \begin{itemize}
            \item Claudio Landim (28 de janeiro de 1965, Rio de Janeiro - RJ) é um matemático brasileiro, que trabalha com probabilidade e suas aplicações à física estatística.
            \item Graduou-se em matemática (1985) pela Pontifícia Universidade Católica do Rio de Janeiro (PUC-Rio). \item Obteve mestrado em estatística e probabilidade (1986) pelo Instituto Nacional de Matemática Pura e Aplicada (IMPA), doutorado em matemática (1990) pela Universidade de Paris Diderot e fez estágio de pós-doutoramento (1994) no Instituto Courant, da Universidade de Nova Iorque, nos EUA. 
        \end{itemize}
    \end{block}
\end{frame}


% \begin{frame}{}
%     \begin{center}
% Imagem 2: Claudio Landim
%         \includegraphics[width=10cm]{landim.jpg}
%     \end{center}
% \end{frame}


\begin{frame}{}
    \begin{block}{Currículo}
        \begin{itemize}
            \item Foi pesquisador na Universidade de Rouen, França (1988–1994), atuou como vice-presidente da Sociedade Brasileira de Matemática (SBM) entre 1999 e 2000 e foi membro do comitê de área na Coordenação de Aperfeiçoamento de Pessoal de Nível Superior – Capes (2005-2007) e da Fundação de Amparo à Pesquisa do Estado do Rio de Janeiro – Faperj (2008-2012). 
            \item É pesquisador do IMPA.  Sua pesquisa concentra-se no campo das ciências exatas e da terra, com ênfase em probabilidade e estatística. 
        \end{itemize}
    \end{block}
\end{frame}


\begin{frame}{}
    \begin{block}{Premios, títulos e participações}
        \begin{itemize}
            \item Dentre prêmios e títulos, destacam-se a medalha de bronze do Centro Nacional de Pesquisa Científica da França (CNRS, na sigla em francês) por suas contribuições à teoria do comportamento hidrodinâmico de sistemas de partículas; 
            \item o Prêmio de Matemática da Academia Mundial de Ciências (TWAS, na sigla em inglês), a Comenda da Ordem Nacional do Mérito Científico (2010) e a Medalha Juscelino Kubitschek (2016). 
            \item Foi palestrante convidado do Congresso Internacional de Matemáticos no Rio de Janeiro (2018: Variational formulae for the capacity induced elliptic differential operators).[3]
            \item Para o Congresso Internacional de Matemáticos de 2022 em São Petersburgo está listado como palestrante convidado.
        \end{itemize}
    \end{block}
\end{frame}


\begin{frame}{}
    \begin{block}{}
        \begin{itemize}
            \item Além da Academia Brasileira de Ciências (ABC), é membro da Academia Mundial de Ciências (TWAS) e da Sociedade Brasileira de Matemática (SBM).
            \item O diretor-adjunto do IMPA, Claudio Landim, é o ganhador do Science Education Prize 2019, que teve como tema desta edição o “Desenvolvimento de Material Científico Educacional”. 
            \item O pesquisador foi escolhido entre candidatos da América Latina e do Caribe por suas contribuições significativas e inovadoras ao material científico educacional, como a criação do Portal do Saber e do Programa OBMEP na Escola. 
            \item Landim é membro da Academia Brasileira de Ciências (ABC) desde 2001 e recebeu  a comenda da Ordem Nacional do Mérito Científico, em 2010.
        \end{itemize}
    \end{block}  
\end{frame}


\begin{frame}{}
    \begin{block}{Principais Obras}
        \begin{itemize}
            \item Comportement Hydrodynamique et Grandes Déviations de Processus à une infinité de Particules (Tese de Mestrado)
            \item Hydrodynamical Equation for Attractive Particle Systems on Z d , Annals of Probability, Volume 19, 1991, p. 1537–1558
            \item com H. T. Yau: Convergence to equilibrium of conservative particle systems on Z d , Annals of Probability, Volume 31, 2003, p. 115–147
            \item com Tomasz Komorowski, Stefano Olla: Fluctuations in Markov Processes, Time Symmetry and Martingale Approximation, Grundlehren der mathematischen Wissenschaften 345, Springer 2012
            \item Hydrodynamic Limits of Interacting Particle Systems, in: ICTP Lecture Notes 17, School and Conference on Probability Theory, Triest 2004, p. 57–100
        \end{itemize}
    \end{block}
\end{frame}

    
% \begin{frame}{REFERÊNCIAS}
%     \begin{itemize}
%         \item Claudio Landim. Academia Brasileira de Sinais (ABC). Disponível em: http://www.abc.org.br/membro/claudio-landim/ . Acesso em 20 de maio de 2023.
%         \item DESCARTES, René. Discurso do Método. Trad. Paulo Neves e introdução de Denis Lerrer Rosenfield. Porto Alegre: L&PM Editores, 2010, p. 37. 
%         \item PORFÍRIO, Francisco. "René Descartes"; Brasil Escola. Disponível em: https://brasilescola.uol.com.br/biografia/rene-descartes.htm. Acesso em 20 de maio de 2023
%         \item SILVA, Marcos Noé Pedro da. "A Matemática de René Descartes (1596 – 1650)"; Brasil Escola. Disponível em: https://brasilescola.uol.com.br/matematica/a-matematica-rene-descartes-15961650.htm . Acesso em 11 de maio de 2023.
%     \end{itemize}
% \end{frame}

\end{document}